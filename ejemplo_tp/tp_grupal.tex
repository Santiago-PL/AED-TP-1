\documentclass[10pt,a4paper]{article}

\input{AEDmacros}
\usepackage{caratula} % Version modificada para usar las macros de algo1 de ~> https://github.com/bcardiff/dc-tex


\titulo{Trabajo Práctico 1}
\subtitulo{Especificación y WP}

\fecha{\today}

\materia{Algoritmos y Estructuras de Datos I}
\grupo{Grupo "gliptodonte24"}

\integrante{Maydana, Daniel}{¿¿¿/¿¿}{email1@dominio.com}
\integrante{Lozada, Jack}{1142/22}{nothingbutjack2200@gmail.com}
\integrante{Cian, Andrés Bautista}{937/21}{andycia802@gmail.com}
\integrante{Perez Lanzillotta, Santiago}{586/16}{santi.perezl@hotmail.com}

\begin{document}

\maketitle



\section{Especificaci'on}

\subsection{redistribucionDeLosFrutos}

%  \  Ejercicio 1

\begin{proc}{redistribucionDeLosFrutos}{\In recursos : \TLista{\float}, \In cooperan : \TLista{\bool}}{\TLista{\float}}
	\requiere{\paraTodo[unalinea]{recurso}{recursos}{recurso \geq 0} \yLuego |recursos| = |cooperan|}
	\asegura {\\
 |res| = |cooperan| \yLuego \\ 
 \paraTodo[unalinea]{i}{\ent} {(0 \leq i \leq |res|)  \implicaLuego \\ (\IfThenElse{cooperan[i] = True}{res[i] = \frac{fondo(recursos,  cooperan)}{|cooperan|}}{res[i] = recursos[i] + \frac{fondo(recursos,cooperan)}{|cooperan|}} }) \\
 }


\vspace{0.50cm}

\aux{fondo}{in recursos : {\TLista{\ent}}, in cooperan : {\TLista{\bool}}}{\ent} {
\sum\limits_{i=0} \limits^{|recursos|-1} 
({\IfThenElse{cooperan[i] \hspace{0.15cm} = \hspace{0.15cm} True}{recursos[i]}{0}})}
    
 % \   {\ifthenelse{\equal{cooperan[i]}{True}}{recursos[i]}{0}}}

%  \   {if cooperan[i] = True then 1 else 0 fi}}

\end{proc}

\vspace{1.2cm}

\subsection{trayectoriaDeLosFrutosIndividualesALargoPlazo}

%  \  Ejercicio 2

\begin{proc}{trayectoriaDeLosFrutosIndividualesALargoPlazo}{\Inout trayectorias : \TLista{\TLista{\float}}, \In cooperan : \TLista{\bool}, \In apuestas : \TLista{\TLista{\float}}, \In pagos : \TLista{\TLista{\float}}, \In eventos : \TLista{\TLista{\bool}}}{\TLista{\TLista{\float}}}
% requiere entradasValidas (misma cantidad de jugadores en todas los parámetros de entrada)
    \requiere{\paraTodo[unalinea]{jugadores}{\nat}{|trayectorias[jugador]| = |apuestas| = |pagos| = |eventos| \yLuego |cooperan| = |trayectorias| = |apuestas[jugador]| = |pagos[jugador]| = |eventos[jugador]|}}
    \asegura{\\
    \paraTodo[unalinea]{jugador}{\nat}{0 \leq jugador < |trayectorias| \paraTodo[unalinea]{i}{\ent}{0 \leq i < |trayectorias[jugador]| - 1\implicaLuego \newline trayectorias[i + 1] = redistribucionDeLosFrutos(trayectorias[i] + ganancias(apuestas[i], pagos[i], eventos[i]), cooperan)}} \\
    }

    \aux {ganancias}{\In apuestas : \TLista{\TLista{\float}}, \In pagos : \TLista{\TLista{\float}}, \In eventos : \TLista{\TLista{\bool}}}{\TLista{\float}} \\ {\paraTodo[unalinea]{individuo}{0 \leq individuo < |apuestas|}{\IfThenElse {apuestas[i] = eventos[i]}{res[i] = pagos[i]}{res[i] = 0}}}
\end{proc}

\vspace{1.2cm}

\subsection{trayectoriaExtrañaEscalera}

%  \  Ejercicio 3

\begin{proc}{trayectoriaExtrañaEscalera}{\In trayectoria : \TLista{\float}}{\bool}
	\requiere{|trayectoria| > 0}
	\asegura{\\
 \sum\limits_{i=0}^{|trayectoria|-1} \IfThenElse{((trayectoria[0] > trayectoria[1]) \vee_L \newline (\paraTodo[unalinea]{j}{1 \leq j < |trayectoria| - 1}{trayectoria[j - 1] < trayectoria[j] < trayectoria[j+1]}) \vee_L (trayectoria[|trayectoria| - 1] > trayectoria[|trayectoria|-2])}{res = 1}{res = 0}) \\
 }
 
\end{proc}

\vspace{1.2cm}

\subsection{individuoDecideSiCooperarONo}

%  \  Ejercicio 4

\begin{proc}{individuoDecideSiCooperarONo}{\In individuo : \nat, \In recursos : \TLista{\float}, \Inout cooperan : \TLista{\bool}, \In apuestas : \TLista{\TLista{\float}}, \In pagos : \TLista{\TLista{\float}}, \In eventos : \TLista{\TLista{\nat}}}{\TLista{\bool}}
	\requiere{0 < individuo < |cooperan| \yLuego |apuestas| = |pagos| = |eventos| \yLuego \paraTodo[unalinea]{jugadores}{\nat}{|cooperan| = |recursos| = |apuestas[jugadores]| = |pagos[jugadores]| = |eventos[jugadores]|} \yLuego 0 \leq individuo < |recursos| \yLuego \paraTodo[unalinea]{recurso}{recursos}{recurso \geq 0}}
	\asegura{True}
\end{proc}


\vspace{1.2cm}



\subsection{individuoActualizaApuesta}

%  \  Ejercicio 5

% \ llama al proc del ejercicio 2, y asegura que el resultado de este ejercicio maximiza la siguiente expresión de la siguiente forma: res \geq trayectorias[individuo[|trayectoria[individuo]|]]

%     \    IDEA:

\begin{proc}{individuoActualizaApuesta}{\In individuo : \nat, \In recursos : \TLista{\float}, \In cooperan : \TLista{\bool}, \Inout apuestas : \TLista{\TLista{\float}}, \In pagos : \TLista{\TLista{\float}}, \In eventos : \TLista{\TLista{\bool}}}{\TLista{\TLista{\float}}}
    \requiere{0 < individuo < |cooperan|}
    \asegura{trayectoriaDeLosFrutosIndividualesALargoPlazo[res] \geq trayectorias[individuo[|trayectoria[individuo]|]]}

\end{proc}

\end{document}
