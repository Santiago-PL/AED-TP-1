\documentclass[10pt,a4paper]{article}

\input{AEDmacros}
\usepackage{caratula} % Version modificada para usar las macros de algo1 de ~> https://github.com/bcardiff/dc-tex


\titulo{Trabajo Práctico 1}
\subtitulo{Especificación y WP}

\fecha{\today}

\materia{Algoritmos y Estructuras de Datos I}
\grupo{Grupo "gliptodonte24"}

\integrante{Maydana, Dani}{001/01}{email1@dominio.com}
\integrante{Apellido, Jack}{002/01}{email2@dominio.com}
\integrante{Cian, Andr'es Bautista}{937/21}{andycia802@gmail.com}
\integrante{Apellido, ElFantasmaDeLaB}{004/01}{email4@dominio.com}
% Pongan cuantos integrantes quieran

% Declaramos donde van a estar las figuras
% No es obligatorio, pero suele ser comodo
\graphicspath{{../static/}}

\begin{document}

\maketitle



\section{Especificaci'on}

\subsection{redistribucionDeLosFrutos}


\begin{proc}{redistribucionDeLosFrutos}{\In recursos : \TLista{\float}, \In cooperan : \TLista{\bool}}{\TLista{\float}}
	\requiere{|recursos| > 0 \yLuego |recursos| = |cooperan| }
	\asegura {\\
 |res| = |cooperan| \yLuego \\ 
 \paraTodo[unalinea]{i}{\ent} {(0 \leq i \leq |res|)  \implicaLuego \\ (\IfThenElse{cooperan[i] = True}{res[i] = \frac{fondo(recursos,  cooperan)}{|cooperan|}}{res[i] = recursos[i] + \frac{fondo(recursos,cooperan)}{|cooperan|}} }) \\
 }


\vspace{0.50cm}

\aux{fondo}{in recursos : {\TLista{\ent}}, in cooperan : {\TLista{\bool}}}{\ent} {
\sum\limits_{i=0} \limits^{|recursos|-1} 
({\IfThenElse{cooperan[i] \hspace{0.15cm} = \hspace{0.15cm} True}{recursos[i]}{0}})}
    
 % \   {\ifthenelse{\equal{cooperan[i]}{True}}{recursos[i]}{0}}}

%  \   {if cooperan[i] = True then 1 else 0 fi}}

\end{proc}

\vspace{1.5cm}

\subsection{trayectoriaExtrañaEscalera}

%  \  Ejercicio 3

\begin{proc}{trayectoriaExtrañaEscalera}{\In trayectoria : \TLista{\float}}{\bool}
	%    \modifica{parametro1, parametro2,..}
	\requiere{|trayectoria| > 0}
	\asegura{\sum\limits_{i=0}^{|trayectoria|-1} \IfThenElse{((trayectoria[0] > trayectoria[1]) \vee_L \newline (\paraTodo[unaLinea]{j}{1 \leq j < |trayectoria| - 1}{trayectoria[j - 1] < trayectoria[j] < trayectoria[j+1]}) \vee_L (trayectoria[|trayectoria| - 1] > trayectoria[|trayectoria|-2])}{res = 1}{res = 0})}
\end{proc}


\end{document}
