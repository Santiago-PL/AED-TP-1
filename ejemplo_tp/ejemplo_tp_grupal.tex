\documentclass[10pt,a4paper]{article}

\input{AEDmacros}
\usepackage{caratula} % Version modificada para usar las macros de algo1 de ~> https://github.com/bcardiff/dc-tex


\titulo{Descripci\'on del tp}
\subtitulo{Subtítulo del tp}

\fecha{\today}

\materia{Materia de la carrera}
\grupo{Grupo 42}

\integrante{Apellido, Nombre1}{001/01}{email1@dominio.com}
\integrante{Apellido, Nombre2}{002/01}{email2@dominio.com}
\integrante{Apellido, Nombre3}{003/01}{email3@dominio.com}
\integrante{Apellido, Nombre4}{004/01}{email4@dominio.com}
% Pongan cuantos integrantes quieran

% Declaramos donde van a estar las figuras
% No es obligatorio, pero suele ser comodo
\graphicspath{{../static/}}

\begin{document}

\maketitle

\section{Ejemplo de sección}
\subsection{Subsección: ambientes comunes de \LaTeX}

Lo principal: las fórmulas. Se puede poner en una linea, como $x_i = x_{i-1} + x_{i-2}$, o ponerse más grande:

\begin{equation}
	\sum\limits_{i=0}^{n} i
	\label{eq:1}
\end{equation}

Y se pueden citar ecuaciones con \verb|\eqref{nombreDeEq}|: \eqref{eq:1}

Ejemplo de itemizado:

\begin{itemize}
	\item Item 1
	\item Item 2
	\item Item 3
\end{itemize}

Ejemplo de enumerado con menor distancia entre items:

\begin{enumerate} \setlength\itemsep{0cm}
	\item Item 1
	\item Item 2
	\item Item 3
\end{enumerate}

Podemos escribir mucho texto. Mucho texto. Mucho texto. Mucho texto. Mucho texto. Mucho texto. Mucho texto. Mucho texto. Mucho texto. Mucho texto. Mucho texto.

Otro párrafo. Otro párrafo. Otro párrafo. Otro párrafo. Otro párrafo. Otro párrafo. Otro párrafo. Otro párrafo. Otro párrafo. Otro párrafo. Otro párrafo. Otro párrafo. Otro párrafo.

\vspace{0.3cm}

Le agregamos una separación entre párrafos. Le agregamos una separación entre párrafos. Le agregamos una separación entre párrafos. Le agregamos una separación entre párrafos. Le agregamos una separación entre párrafos.

\vspace{0.3cm}

La tabla \ref{tab:ejemplo} es un ejemplo de cómo se hace una tabla.

\begin{table}[h!]
	\centering
	\begin{tabular}{||l c c r||} 
		\hline
		Col1 & Col2 & Col2 & Col3 \\ [0.5ex] 
		\hline\hline
		1 & 6 & 87837 & 787 \\ 
		2 & 7 & 78 & 5415 \\
		3 & 545 & 778 & 7507 \\
		4 & 545 & 18744 & 7560 \\
		5 & 88 & 788 & 6344 \\
		\hline
	\end{tabular}
	\caption{Ejemplo de tabla}
	\label{tab:ejemplo}
\end{table}


La figura \ref{fig:subfigs} es un ejemplo de cómo se agrega una imagen.

\begin{figure}[ht]
	\centering
	\includegraphics[width=0.6\textwidth]{logo_dc.jpg}
	\caption{Ejemplo de figura}
	\label{fig:ejemplo}
\end{figure}

\begin{figure}[ht!]
	\begin{subfigure}{0.5\textwidth}
		\includegraphics[width=0.9\linewidth]{LaTeX-project} 
		\caption{Logo de LaTeX}
		\label{fig:subfig1}
	\end{subfigure}
	\begin{subfigure}{0.5\textwidth}
		\includegraphics[width=0.7\linewidth]{TeX}
		\caption{Logo de TeX}
		\label{fig:subfig2}
	\end{subfigure}
	\caption{Ejemplo para poner dos figuras juntas. Y citarlas por separado a (\subref{fig:subfig1}) y (\subref{fig:subfig2}).}
	% OJO: el caption siempre va antes del label
	\label{fig:subfigs}
\end{figure}



% Para hacer que quede todo en una misma linea, se puede usar minipage
%\begin{minipage}[t]{\textwidth}
	\begin{lstlisting}[caption={Ejemplo de código (usando los estilos de la cátedra, ver las macros para más detalles)},label=code:for]
res := 0;
i := 0;
while (i < s.size()) do
	res := res + s[i];
	i := i + 1
endwhile
	\end{lstlisting}
%\end{minipage}

Si se pone un label al \verb|lstlisting|, se puede referenciar: Código \ref{code:for}.


\subsection{Nuestro TP}

Ejercicio 1

\begin{proc}{redistribucionDeLosFrutos}{\In recursos : \TLista{\float}, \In cooperan : \TLista{\bool}}{\TLista{\float}}
	%    \modifica{parametro1, parametro2,..}
	\requiere{recursos > 0}
	\asegura{recursos[los \hspace{0.15cm} que \hspace{0.15cm} SI \hspace{0.15cm} cooperan] \hspace{0.15cm} = \hspace{0.15cm} fondo \hspace{0.15cm} / \hspace{0.15cm} sumaDeLosQueCooperan(cooperan)}
    \asegura{recursos[los \hspace{0.15cm} que \hspace{0.15cm} NO \hspace{0.15cm} cooperan] \hspace{0.15cm} = \hspace{0.15cm} old(los que NO cooperan) + fondo \hspace{0.15cm} / \hspace{0.15cm} \neg sumaDeLosQueCooperan}
	\aux{fondo}{in: recursos}{\nat}{\paraTodo[unaLinea]{i}{cooperan}
    {if \hspace{0.15cm} cooperan[i] \hspace{0.15cm} = \hspace{0.15cm} True \hspace{0.15cm} then \hspace{0.15cm} recursos[i] \hspace{0.15cm} else \hspace{0.15cm} 0 fi}}
    \aux{sumaDeLosQueCooperan}{in: cooperan}{seq \ent}{\paraTodo[unaLinea]{i}{cooperan}{\IfThenElse{cooperan[i]=True}{recursos[i]}{0}}}
    
 % \   {\ifthenelse{\equal{cooperan[i]}{True}}{recursos[i]}{0}}}

%  \   {if cooperan[i] = True then 1 else 0 fi}}

\end{proc}

\subsection{Macros de la cátedra para especificar}

\begin{proc}{nombre}{\In paramIn : \nat, \Inout paramInout : \TLista{\ent}}{tipoRes}
	%    \modifica{parametro1, parametro2,..}
	\requiere{expresionBooleana1}
	\asegura{expresionBooleana2}
	\aux{auxiliar1}{parametros}{tipoRes}{expresion}
	\pred{pred1}{parametros}{expresion} 
\end{proc}

\aux{auxiliarSuelto}{parametros}{tipoRes}{expresion}
% \paraTodo{variable}{tipo}{expresion}
% \existe{variable}{tipo}{expresion}
% Pueden tener [unalinea] para que no se divida en varias lineas
\pred{predSuelto}{parametros}{\paraTodo[unalinea]{variable}{tipo}{algo \implicaLuego expresion}}
\pred{predSuelto}{parametros}{\existe[unalinea]{variable}{tipo}{algo \yLuego expresion}}



\end{document}
